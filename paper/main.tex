\documentclass[sigconf]{acmart}



\usepackage{xspace} % helps spaces be correct when not using {} with commands

% Data collection timeframe
\newcommand{\nDataStartDate}{07/30/XXXX\xspace}
\newcommand{\nDataEndDate}{05/04/XXXX\xspace}
\usepackage{caption}
\usepackage{subcaption}
\usepackage[ruled,vlined,linesnumbered]{algorithm2e}

%%
%% \BibTeX command to typeset BibTeX logo in the docs
\AtBeginDocument{%
  \providecommand\BibTeX{{%
    \normalfont B\kern-0.5em{\scshape i\kern-0.25em b}\kern-0.8em\TeX}}}

%% Rights management information.  This information is sent to you
%% when you complete the rights form.  These commands have SAMPLE
%% values in them; it is your responsibility as an author to replace
%% the commands and values with those provided to you when you
%% complete the rights form.
\copyrightyear{2021}
\acmYear{2021}
\setcopyright{iw3c2w3}
\acmConference[WWW '21]{Proceedings of the Web Conference 2021}{April 19--23, 2021}{Ljubljana, Slovenia}
\acmBooktitle{Proceedings of the Web Conference 2021 (WWW '21), April 19--23, 2021, Ljubljana, Slovenia}
\acmPrice{}
\acmDOI{10.1145/3442381.3449824}
\acmISBN{978-1-4503-8312-7/21/04}

\settopmatter{printacmref=true}

%%
%% Submission ID.
%% Use this when submitting an article to a sponsored event. You'll
%% receive a unique submission ID from the organizers
%% of the event, and this ID should be used as the parameter to this command.
%%\acmSubmissionID{123-A56-BU3}

%%
%% The majority of ACM publications use numbered citations and
%% references.  The command \citestyle{authoryear} switches to the
%% "author year" style.
%%
%% If you are preparing content for an event
%% sponsored by ACM SIGGRAPH, you must use the "author year" style of
%% citations and references.
%% Uncommenting
%% the next command will enable that style.
%%\citestyle{acmauthoryear}

%%
%% end of the preamble, start of the body of the document source.
\begin{document}

%%
%% The "title" command has an optional parameter,
%% allowing the author to define a "short title" to be used in page headers.
\title{Measuring a Blockchain-based Metaverse: Investigating Decentraland}

%%
%% The "author" command and its associated commands are used to define
%% the authors and their affiliations.
%% Of note is the shared affiliation of the first two authors, and the
%% "authornote" and "authornotemark" commands
%% used to denote shared contribution to the research.

\author{Friedhelm Victor}
\affiliation{%
  \institution{Technische Universit{\"a}t Berlin}
  \streetaddress{Straße des 17. Juni 135, 10623 Berlin}
  \city{Berlin}
  \country{Germany}}
\email{friedhelm.victor@tu-berlin.de}


\begin{abstract}
  \input{sections/00_abstract.tex}
\end{abstract}

\begin{CCSXML}
<ccs2012>
<concept>
<concept_id>10010405.10003550.10003551</concept_id>
<concept_desc>Applied computing~Digital cash</concept_desc>
<concept_significance>500</concept_significance>
</concept>
<concept>
<concept_id>10010405.10003550.10003554</concept_id>
<concept_desc>Applied computing~Electronic funds transfer</concept_desc>
<concept_significance>300</concept_significance>
</concept>
<concept>
<concept_id>10010405.10010462</concept_id>
<concept_desc>Applied computing~Computer forensics</concept_desc>
<concept_significance>100</concept_significance>
</concept>
</ccs2012>
\end{CCSXML}


\ccsdesc[500]{Applied computing~Digital cash}
\ccsdesc[300]{Applied computing~Electronic funds transfer}
\ccsdesc[100]{Applied computing~Computer forensics}

\keywords{wash trading, self-trades, measurement, crime, blockchain}

\maketitle

\section{Introduction}
At the latest since the announcement that Facebook has changed its name to Meta, the \textit{Metaverse} topic has received a lot of coverage. Throughout 2022, articles about it were published in newspapers like the New York Times, Washington Post and many others.
But the topic is not entirely new. The idea of a metaverse was originally described in 1992 in the book Snow Crash by Neal Stephenson. At the time, he described it as a virtual universe parallel to the real world. Since then, virtual worlds in which users can interact with each other have become a well established concept since then. They are particularly common in games, and have received significant research attention. How the Metaverse will look like in the future and on which technologies it will rely, however, is still open. VR or AR headsets could play a major role here, 

One particular domain with 

Lorem ipsum dolor sit amet, consetetur sadipscing elitr, sed diam nonumy eirmod tempor invidunt ut labore et dolore magna aliquyam erat, sed diam voluptua. At vero eos et accusam et justo duo dolores et ea rebum. Stet clita kasd gubergren, no sea takimata sanctus est Lorem ipsum dolor sit amet. Lorem ipsum dolor sit amet, consetetur sadipscing elitr, sed diam nonumy eirmod tempor invidunt ut labore et dolore magna aliquyam erat, sed diam voluptua. At vero eos et accusam et justo duo dolores et ea rebum. Stet clita kasd gubergren, no sea takimata sanctus est Lorem ipsum dolor sit amet. Lorem ipsum dolor sit amet, consetetur sadipscing elitr, sed diam nonumy eirmod tempor invidunt ut labore et dolore magna aliquyam erat, sed diam voluptua. At vero eos et accusam et justo duo dolores et ea rebum. Stet clita kasd gubergren, no sea takimata sanctus est Lorem ipsum dolor sit amet.   

Duis autem vel eum iriure dolor in hendrerit in vulputate velit esse molestie consequat, vel illum dolore eu feugiat nulla facilisis at vero eros et accumsan et iusto odio dignissim qui blandit praesent luptatum zzril delenit augue duis dolore te feugait nulla facilisi. Lorem ipsum dolor sit amet, consectetuer adipiscing elit, sed diam nonummy nibh euismod tincidunt ut laoreet dolore magna aliquam erat volutpat.   

Ut wisi enim ad minim veniam, quis nostrud exerci tation ullamcorper suscipit lobortis nisl ut aliquip ex ea commodo consequat. Duis autem vel eum iriure dolor in hendrerit in vulputate velit esse molestie consequat, vel illum dolore eu feugiat nulla facilisis at vero eros et accumsan et iusto odio dignissim qui blandit praesent luptatum zzril delenit augue duis dolore te feugait nulla facilisi.   

Nam liber tempor cum soluta nobis eleifend option congue nihil imperdiet doming id quod mazim placerat facer possim assum. Lorem ipsum dolor sit amet, consectetuer adipiscing elit, sed diam nonummy nibh euismod tincidunt ut laoreet dolore magna aliquam erat volutpat. Ut wisi enim ad minim veniam, quis nostrud exerci tation ullamcorper suscipit lobortis nisl ut aliquip ex ea commodo consequat.   

Duis autem vel eum iriure dolor in hendrerit in vulputate velit esse molestie consequat, vel illum dolore eu feugiat nulla facilisis.   

At vero eos et accusam et justo duo dolores et ea rebum. Stet clita kasd gubergren, no sea takimata sanctus est Lorem ipsum dolor sit amet. Lorem ipsum dolor sit amet, consetetur sadipscing elitr, sed diam nonumy eirmod tempor invidunt ut labore et dolore magna aliquyam erat, sed diam voluptua. At vero eos et accusam et justo duo dolores et ea rebum. Stet clita kasd gubergren, no sea takimata sanctus est Lorem ipsum dolor sit amet. Lorem ipsum dolor sit amet, consetetur sadipscing elitr, At accusam aliquyam diam diam dolore dolores duo eirmod eos erat, et nonumy sed tempor et et invidunt justo labore Stet clita ea et gubergren, kasd magna no rebum. sanctus sea sed takimata ut vero voluptua. est Lorem ipsum dolor sit amet. Lorem ipsum dolor sit amet, consetetur sadipscing elitr, sed diam nonumy eirmod tempor invidunt ut labore et dolore magna aliquyam erat.   

Consetetur sadipscing elitr, sed diam nonumy eirmod tempor invidunt ut labore et dolore magna aliquyam erat, sed diam voluptua. At vero eos et accusam et justo duo dolores et ea rebum. Stet clita kasd gubergren, no sea takimata sanctus est Lorem ipsum dolor sit amet. Lorem ipsum dolor sit amet, consetetur sadipscing elitr, sed diam nonumy eirmod tempor invidunt ut labore et dolore magna aliquyam erat, sed diam voluptua. At vero eos et accusam et justo duo dolores et ea rebum. Stet clita

Our contributions can be summarized as follows:
\begin{itemize}
    \item First contribution Lorem ipsum dolor sit amet, consetetur sadipscing elitr, sed diam nonumy eirmod tempor
    \item second contribution Lorem ipsum dolor sit amet, consetetur sadipscing elitr, sed diam nonumy eirmod tempor
\end{itemize}

\newpage
\input{sections/02_background.tex}
\input{sections/03_related_work.tex}
\section{Data Description}
We have collected data from multiple sources. The original starting point is live user activity from Decentraland servers, which can then be enriched with on-chain transaction information from Polygon and Ethereum.

\subsection{Decentraland activity data}

\subsection{On-chain transaction data}

\bibliographystyle{ACM-Reference-Format}
\bibliography{main}

\end{document}
\endinput
